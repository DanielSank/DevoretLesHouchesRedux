\section{Quantum mechanics of linear dissipative circuits}

\subsection{Quantum description of electrical circuits}

\subsection{Useful relations}

\subsection{The quantum LC oscillator}

\begin{itemize}

\item Page 373: I believe the statement just after Equation (3.21) should read ``...the high temperature $(k_bT \gg \hbar \omega_0)$ result $\langle \phi^2 \rangle=k_b T L$.''
The quantity as written, $k_b T C$ has dimensions of charge squared.

\end{itemize}

\subsection{The quantum fluctuation-dissipation theorem}

\begin{itemize}

\item Page 374: The use of square brackets in $Z[\omega]$ in equations (3.25), (3.27), (3.29) and (3.30) is not clear. From the previous parts of the course it seems the square brackets are used only when talking about the generalized impedance, but these equations involve the usual impedance. It may be that the square bracket is intended for \emph{any} frequency function, in which case my previous note about this issue should be interpreted as a small inconsistency at the earlier part of the course.

\item Page 375: There seems to be a factor of 2 missing in equation (3.30). If I take equation (3.29) for $\hbar \omega > k_b T$, the $\coth$ is asymptotically equal to unity, giving $2 \hbar \omega \Re Z(\omega)$.

\end{itemize}

\subsection{Interpretation of the quantum spectral density}

\begin{itemize}

\item A reference to http://arxiv.org/abs/cond-mat/0210247 would be helpful here. There are a large number of papers which say that the quantum spectral density at positive and negative frequencies ``corresponds'' or ``is related'' to emission and absorption, but do not actually demonstrate this. The referenced work gives a very clear and brief demonstration of \emph{why} this correspondence arises in a mathematical language accessible even for undergraduate students.

\end{itemize}

\subsection{Quantum fluctuations in the damped $LC$ oscillator}

I did not check this section.

\subsection{Low temperature limit}

\begin{itemize}

\item I think this subsection was intended to be a subsubsection.

\end{itemize}

