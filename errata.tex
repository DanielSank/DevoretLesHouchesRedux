\documentclass{article}

\usepackage{amssymb}
\usepackage{amsmath}
\usepackage{amstext}

\newcommand{\ket}[1]{|#1 \rangle}
\newcommand{\bbraket}[3]{\langle #1 | #2 | #3 \rangle}

\title{Errata and suggested edits for \\
Quantum Fluctuations in Electrical Circuits}

\author{Daniel Sank}

\begin{document}

\maketitle

\section{Introduction}

\begin{itemize}

\item Page 356: the formula for the quality factor of the damped resonator should be
\begin{equation}
Q = \frac{1}{\Re{Y(\omega_0)} Z_0} \nonumber \, .
\end{equation} 
The real part is omitted in the text.

\item Page 357: In the caption of Fig. 2, the phrase "effective inductance of the junction" might be changed to "small-current inductance of the junction".

\item Page 358: The term "effective inductance" for the junction is introduced. It would be nice to move this prior to references to Fig. 2.

\end{itemize}


\section{Hamiltonian description of the classical dynamics of electrical circuits}

\begin{itemize}

\item Page 361: I think "An Hamiltonian" should be "A Hamiltonian".

\item Page 363: There are many occurrences of the expression $2e / \hbar$ throughout the course. I wonder if using $\Phi_0 \equiv h / 2e$ would be easier to read.

\item Page 363: It would be convenient to label $C_1$, $C_2$, etc. in Figure 5.

\item Page 364: I think equations (2.14) and (2.15) should be
\begin{align}
C_1 \ddot{\phi}_a + C_3 ( \ddot{\phi}_a - \ddot{\phi}_b) &= - \frac{\phi_a}{L_1} - \frac{\phi_a - \phi_b + \tilde{\Phi}}{L_3} \nonumber \\
C_2 \ddot{\phi}_b + C_3 ( \ddot{\phi}_b - \ddot{\phi}_a) &= - \frac{\phi_b}{L_2} - \frac{\phi_b - \phi_a - \tilde{\Phi}}{L_3} \nonumber \, .
\end{align}
Note that the Lagrangian (2.16) is correct, but does not match (2.14) and (2.15).

\item Page 365: I believe (2.18) should be
\begin{align}
H = \frac{1}{C_1 C_2 + C_1 C_3 + C_2 C_3} & \left[
\frac{(C_2 + C_3) q_a^2}{2}
+ \frac{(C_1 + C_3)q_b^2}{2}
+ C_3 q_a q_b \right] \nonumber \\
+ & \left[
\frac{\phi_a^2}{2 L_1} + \frac{\phi_b^2}{2 L_2} + \frac{(\phi_a - \phi_b + \tilde{\Phi})^2}{2 L_3} \right] \nonumber \, .
\end{align}
There is an easy way to see this.
The kinetic part of the Lagrangian can be written
\begin{equation*}
\mathcal{L}_{T} = \frac{1}{2} \dot{\Phi}^\top M \dot{\Phi}
\end{equation*}
where $M$ is represented as
\begin{equation*}
M = \left[ \begin{array} {cc} C_1 + C_3 & -C_3 \\ -C_3 & C_2 + C_3 \end{array} \right]
\end{equation*}
and
\begin{equation*}
\dot{\Phi} = \left( \begin{array}{c} \dot{\phi_a} \\ \dot{\phi}_b \end{array} \right)
\end{equation*}
Writing out the the canonical momenta $q_a$ and $q_b$ we find
\begin{equation*}
Q = M \dot{\Phi}
\end{equation*}
where here
\begin{equation*}
Q = \left( \begin{array}{c} q_a \\ q_b \end{array} \right) \, .
\end{equation*}
Therefore,
\begin{equation*}
\mathcal{L}_T = \frac{1}{2} Q^\top M^{-1} Q
\end{equation*}
and so the kinetic part of the Hamiltonian is
\begin{equation}
H = Q^\top \dot{\Phi} - \mathcal{L}_T = \frac{1}{2} Q^\top M^{-1} Q \label{eq:Hamiltonian} \, .
\end{equation}
The inverse matrix is
\begin{equation*}
M^{-1} = \frac{1}{C_1 C_2 + C_1 C_3 + C_2 C_3}
\left[ \begin{array}{cc} C_2 + C_3  & C_3 \\ C_3 & C_1 + C_2 \end{array} \right] \, .
\end{equation*}
Expanding Eq.\,(\ref{eq:Hamiltonian}) with this matrix gives the result I propose.

\item Page 365: In equation (2.18) it may be very helpful to point out that the expression
\begin{equation}
\left( \frac{C_2 + C_3}{C_1 C_2 + C_1 C_3 + C_2 C_3} \right)^{-1}
\end{equation}
is just the capacitance to ground of node $a$, and similarly for node $b$.
Therefore, the Hamiltonian term
\begin{equation}
\frac{1}{C_1 C_2 + C_1 C_3 + C_2 C_3} \frac{(C_2 + C_3)q_a^2}{2}
\end{equation}
can be interpreted as
\begin{equation}
\frac{q_a^2}{2 C'}
\end{equation}
where $C'$ is the capacitance to ground of the node.

\end{itemize}

\end{document}