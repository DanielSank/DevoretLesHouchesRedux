\section{Hamiltonian description of the classical dynamics of electrical circuits}

\subsection{Non-dissipative circuits}

\subsubsection{Branch variables}

\begin{itemize}

\item Page 361: I think "An Hamiltonian" should be "A Hamiltonian".

\end{itemize}

\subsubsection{The degrees of freedom of a circuit}

\begin{itemize}

\item Page 363: There are many occurrences of the expression $2e / \hbar$ throughout the course. I wonder if using $\Phi_0 \equiv h / 2e$ would be easier to read.

\item Page 363: It would be convenient to label $C_1$, $C_2$, etc. in Figure 5.

\end{itemize}

\subsubsection{Lagrangian of a circuit}

\begin{itemize}

\item Page 364: I think equations (2.14) and (2.15) should be
\begin{align}
C_1 \ddot{\phi}_a + C_3 ( \ddot{\phi}_a - \ddot{\phi}_b) &= - \frac{\phi_a}{L_1} - \frac{\phi_a - \phi_b + \tilde{\Phi}}{L_3} \nonumber \\
C_2 \ddot{\phi}_b + C_3 ( \ddot{\phi}_b - \ddot{\phi}_a) &= - \frac{\phi_b}{L_2} - \frac{\phi_b - \phi_a - \tilde{\Phi}}{L_3} \nonumber \, .
\end{align}
Note that the Lagrangian (2.16) is correct, but does not match (2.14) and (2.15).

\end{itemize}

\subsubsection{Node charges: the conjugate momenta of node fluxes}

\subsubsection{Hamiltonian of a circuit}

\begin{itemize}

\item Page 365: I believe (2.18) should be
\begin{align}
H = \frac{1}{C_1 C_2 + C_1 C_3 + C_2 C_3} & \left[
\frac{(C_2 + C_3) q_a^2}{2}
+ \frac{(C_1 + C_3)q_b^2}{2}
+ C_3 q_a q_b \right] \nonumber \\
+ & \left[
\frac{\phi_a^2}{2 L_1} + \frac{\phi_b^2}{2 L_2} + \frac{(\phi_a - \phi_b + \tilde{\Phi})^2}{2 L_3} \right] \nonumber \, .
\end{align}
%There is an easy way to see this.
%The kinetic part of the Lagrangian can be written
%\begin{equation}
%\mathcal{L}_{T} = \frac{1}{2} \dot{\Phi}^\top M \dot{\Phi}
%\end{equation}
%where $M$ is represented as
%\begin{equation}
%M = \left[ \begin{array} {cc} C_1 + C_3 & -C_3 \\ -C_3 & C_2 + C_3 \end{array} \right]
%\end{equation}
%and
%\begin{equation}
%\dot{\Phi} = \left( \begin{array}{c} \dot{\phi_a} \\ \dot{\phi}_b \end{array} \right)
%\end{equation}
%Writing out the the canonical momenta $q_a$ and $q_b$ we find
%\begin{equation}
%Q = M \dot{\Phi}
%\end{equation}
%where here
%\begin{equation}
%Q = \left( \begin{array}{c} q_a \\ q_b \end{array} \right) \, .
%\end{equation}
%Therefore,
%\begin{equation}
%\mathcal{L}_T = \frac{1}{2} Q^\top M^{-1} Q
%\end{equation}
%and so the kinetic part of the Hamiltonian is
%\begin{equation}
%H = Q^\top \dot{\Phi} - \mathcal{L}_T = \frac{1}{2} Q^\top M^{-1} Q \label{eq:Hamiltonian} \, .
%\end{equation}
%The inverse matrix is
%\begin{equation}
%M^{-1} = \frac{1}{C_1 C_2 + C_1 C_3 + C_2 C_3}
%\left[ \begin{array}{cc} C_2 + C_3  & C_3 \\ C_3 & C_1 + C_2 \end{array} \right] \, .
%\end{equation}
%Expanding Eq.\,(\ref{eq:Hamiltonian}) with this matrix gives the result I propose.

\item Page 365: In equation (2.18) it may be very helpful to point out that the expression
\begin{equation}
\left( \frac{C_2 + C_3}{C_1 C_2 + C_1 C_3 + C_2 C_3} \right)^{-1}
\end{equation}
is just the capacitance to ground of node $a$, and similarly for node $b$.
Therefore, the Hamiltonian term
\begin{equation}
\frac{1}{C_1 C_2 + C_1 C_3 + C_2 C_3} \frac{(C_2 + C_3)q_a^2}{2}
\end{equation}
can be interpreted as
\begin{equation}
\frac{q_a^2}{2 C'}
\end{equation}
where $C'$ is the capacitance to ground of the node.

\end{itemize}

\subsubsection{Mechanical Analogy}

\subsubsection{Fields to circuits, circuits to fields}

\subsection{Circuits with linear dissipative elements}

\subsubsection{The Caldeira-Leggett model}

\begin{itemize}

\item Page 368: Just after (2.26) it may be useful to write $Y(t<0)=0$ to make clear the meaning of ``causal''.

\item Page 368: Equation (2.28) is obvious to those who know the Sokhotskyi-Plemelj formula, but on my first time reading the course I did not know it! A simple statement
\begin{equation}
\int \frac{f(x)}{x \pm i \epsilon} \, dx =
\mp i \pi f(0) + \mathcal{P} \int \frac{f(x)}{x}\, dx
\end{equation}
may be appropriate.

\item Page 368: Equations (2.29) through (2.35) should be
\begin{align*}
\omega_{m \neq 0} &= m \Delta \omega \quad \text{(no change)} \\
y_{m \neq 0} &= \frac{2}{\pi m} \Re Y(\omega_m) \\
L_0 &= \frac{1}{\lim_{\omega \rightarrow 0} j \omega Y(\omega)} \quad \text{(I don't understand this equation)} \\
L_{m \neq 0} &= \frac{1}{\omega_m y_m} \quad \text{(no change)} \\
C_{M \neq 0} &= \frac{y_m}{\omega_m} \quad \text{(no change)} \\
Y_m [\omega + i \eta] &= \left[ j C_m(\omega + i \eta) + \frac{1}{j L_m (\omega + i \eta)} \right] ^{-1} \\
Y[\omega + i \eta] &= \frac{i}{L_0(\omega + i \eta)} + \lim_{\Delta \omega \rightarrow 0}
\sum_{m=1}^\infty Y_m(\omega + i \eta); \quad \eta > 0 \quad \text{(no change)}
\end{align*}

\item Page 368: In this discussion it would seem very helpful to mention that adding $i \eta$ to the frequency is absolutely equivalent to adding some resistance to the oscillators (finite $Q$) and then taking the resistances to zero at the end (take $Q$ to infinity). This may seem a more physical way to understand the mathematical prescription.

\end{itemize}

\subsubsection{Voltage and current sources}

\subsubsection{The classical fluctuation-dissipation theorem}

\begin{itemize}

\item Page 370: In equation (2.41) the $d\omega$ should be $dt$.

\item Page 370: In equation (2.42) it seems we should write $Y(\omega)$ instead of $Y[\omega]$ if I understand the conventions correctly. My understanding is that the square brackets are used for the generalized admittance which includes the $i\eta$ shift in the poles, whereas the parentheses are used for the normal admittance.

\end{itemize}
